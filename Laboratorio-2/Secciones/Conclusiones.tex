\section{Conclusiones}\label{sec:conclusiones}

Al alterar de manera que vaya aumentando el factor roll-off, se demuestra que el ancho de banda será superior en comparación al pulso rectangular, esto provoca que se requiera abarcar más en el espectro de frecuencia con el fin de transmitir la señal y disminuir la interferencia intersimbólica.

A pesar que al usar un roll-off de 1 (que es el máximo que se puede ocupar)  ocupe el doble  de ancho de banda mínimo, esto lo solventa al poder solucionar el problema denominado $"$jitter$"$, permitiendo sincronizar la señal recibida en el instante 0.

Otra forma de afectar el ancho de banda del pulso coseno alzado es modificar la frecuencia de muestreo, como se vio en el apartado de $"$ análisis de resultados$"$, al modificar significativamente la frecuencia de muestreo, el pulso tiende a perder una cantidad considerable de muestras y cambiando su forma abruptamente pasando a restringir el ancho de banda de dicha señal.

Finalmente si se ocupa un filtro $"$ AWGN$"$ y se  aplica ruido al diagrama de ojo, esto provocará que el diagrama de ojo posea una notable cantidad de interferencia intersimbólica provocando un error en los flujos de bits en el diagrama de ojos.

