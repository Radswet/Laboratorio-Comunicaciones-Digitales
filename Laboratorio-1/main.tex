\documentclass[lettersize,journal]{IEEEtran}
\usepackage{amsmath,amsfonts}
\usepackage{algorithmic}
\usepackage{algorithm}
\usepackage{array}
\usepackage[caption=false,font=normalsize,labelfont=sf,textfont=sf]{subfig}
\usepackage{textcomp}
\usepackage{stfloats}
\usepackage{url}
\usepackage{verbatim}
\usepackage{graphicx}

\usepackage{cite}
\usepackage{import}
\usepackage{xcolor,listings}
\usepackage{textcomp}
\usepackage{comment}
\usepackage{float}
\usepackage{amsmath}

\lstset{upquote=true}
\definecolor{gray97}{gray}{.97}

\lstdefinelanguage{Mat}{
    frame=Ltb,
    framerule=0pt,
    aboveskip=0.5cm,
    framextopmargin=3pt,
    framexbottommargin=3pt,
    framexleftmargin=0.4cm,
    framesep=0pt,
    rulesep=.4pt,
    backgroundcolor=\color{gray97},
    rulesepcolor=\color{black},
    %
    stringstyle=\ttfamily,
    showstringspaces = false,
    basicstyle=\small\ttfamily,
    commentstyle=\color{gray45},
    keywordstyle=\bfseries,
    %
    numbers=left,
    numbersep=15pt,
    numberstyle=\tiny,
    numberfirstline = false,
    breaklines=true
}

\hyphenation{op-tical net-works semi-conduc-tor IEEE-Xplore}
% updated with editorial comments 8/9/2021

\begin{document}

\title{Laboratorio 1: PAM y PCM}

\author{Felipe Ulloa 1, felipe.ulloa1@mail.udp.cl \\
Alex Parada 2, email alex.parada@mail.udp.cl \\
Fernando Vergara 3, fernando.vergara1@mail.udp.cl \\

Escuela de Informática y Telecomunicaciones \\ \IEEEmembership{Universidad Diego Portales}
        % <-this % stops a space
%\thanks{This paper was produced by the IEEE Publication Technology Group. They are in Piscataway, NJ.}% <-this % stops a space
%\thanks{Manuscript received April 19, 2021; revised August 16, 2021.}
}

% The paper headers
%\markboth{CIT2111: Comunicaciones Digitales, Semestre I 2022}%
%{Shell \MakeLowercase{\textit{et al.}}: A Sample Article Using IEEEtran.cls for IEEE Journals}

%\IEEEpubid{0000--0000/00\$00.00~\copyright~2021 IEEE}
% Remember, if you use this you must call \IEEEpubidadjcol in the second
% column for its text to clear the IEEEpubid mark.

\maketitle

\import{Secciones/}{Introduccion.tex}

\import{Secciones/}{Metodologia.tex}

\import{Secciones/}{Resultados.tex}

\import{Secciones/}{Conclusiones.tex}

\onecolumn
\import{Secciones/}{Anexos.tex}



\nocite{IEEEhowto:kopka}

\bibliographystyle{plain} % We choose the "plain" reference style
\bibliography{refs} % Entries are in the refs.bib file



\end{document}

