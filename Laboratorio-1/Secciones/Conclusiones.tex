\section{Conclusiones}\label{sec:conclusiones}

Al transmitir una señal PAM se puede dar cuenta de un aumento sustancial en el ancho de banda, pero esto lleva consigo una perdida de datos y precisión en la señal modulada, se puede decir que la calidad de la señal se ve disminuida. Esta perdida de datos se puede suplir aumentando los niveles o bits de la PAM por lo que podemos aumentar la precisión cuanto se desee.

Para el apartado PCM se realizó la practica de muestreo, cuantificación y codificación que permitió convertir una señal analógica a una señal digital, al transcurrir la realización de la actividad se  evidencio además que se genera un pequeño  pero irreparable error de  cuantificación.Este error a pesar de ser irreparable se puede hacerse adecuadamente pequeña utilizando una cantidad mayoritaria de bits.

Finalmente de manera comparativa, se puede afirmar que a pesar de que la digitalización de señales permite abarcar más utilidades, esta posee un coste mayoritario en implementación y ancho de banda lo que impediría para algunas funciones no sea tan beneficioso digitalizar.

%luego de esta experiencia podemos concluir que se gana ancho de banda y se pierden datos precisión en la señal y este se ve aumentado a medida que los niveles van aumentando
