\section{Introducción}\label{sec:introduccion}

La Modulación por impulsos codificados (PCM) es un procedimiento que se utiliza para transformar una señal análoga a una señal digital. 

Por otra parte la modulación por amplitud de pulsos (PAM) es un termino que describe la conversión de la señal analógica a una señal del tipo de pulso en el cual la amplitud del pulso representa la información analógica \cite{PAM:Couch}. Una de las características de la PAM es que su ancho de banda es mayor que el de la forma de onda analógica es por esto que se utiliza en distintas areas principalmente en la comunicación de datos como por ejemplo para los estandares De Gigabit Ethernet 100BASE-T1 se utiliza la codificación PAM-3 (Modulación de 3 niveles)o para 100BASE-T2 PAM-5 (Modulación de 5 niveles), o bien para la trasmisión de datos en DRAMs esto debido al constante incremento del ancho de banda de estas.  \cite{electronics10151768}. 

Para esta experiencia se abordaran tanto la PCM como la PAM y sus distintos tipos de muestreos analizando características de estas entre ellas el error de cuantificación y niveles de la pam .